% chercher des documents LaTeX dans styles, corps et bib
\makeatletter\def\input@path{{styles/}{body/}{bib/}}\makeatother

% Utiliser le style rapport.cls
\documentclass[a4paper]{article}

\usepackage[a4paper,vdivide={*,22cm,4cm}]{geometry}
%\usepackage[french]{babel} % style francais
\usepackage{pageGardeEnsta}
\usepackage{lmodern}
\usepackage[T1]{fontenc}
\usepackage[utf8]{inputenc}
\usepackage{indentfirst}

%============ insert code into latex ==============
\usepackage{listings}
\usepackage{color}

\definecolor{dkgreen}{rgb}{0,0.6,0}
\definecolor{gray}{rgb}{0.5,0.5,0.5}
\definecolor{mauve}{rgb}{0.58,0,0.82}

\lstset{frame=tb,
  language=Ruby,
  aboveskip=3mm,
  belowskip=3mm,
  showstringspaces=false,
  columns=flexible,
  basicstyle={\small\ttfamily},
  numbers=none,
  numberstyle=\tiny\color{gray},
  keywordstyle=\color{blue},
  commentstyle=\color{dkgreen},
  stringstyle=\color{mauve},
  breaklines=true,
  breakatwhitespace=true,
  tabsize=3
}
%==================================================
\usepackage{textcomp}
\usepackage{amsmath}

% pour charger des images
\usepackage{graphicx}
% repertoire dans lequel trouver les images
\graphicspath{{imgs/}}
\DeclareGraphicsExtensions{.eps,.ps,.jpg,.bmp,.png,.pdf}
% liens hypertexte dans le document
\usepackage[colorlinks,breaklinks,linkcolor=blue]{hyperref}

\usepackage{pdfpages}

\title{Status report}
\author{ZHENG \textsc{Tao} \& GUO \textsc{Xinrui}\\
  \texttt{tao.zheng@ensta-bretagne.org}\\
  \texttt{Système logiciel et Sécurité}\\}
\date{\today}
\doctype{Rapport}
\promo{CI 2016}
\etablissement{\textsc{Ensta} Bretagne\\2, rue François Verny\\
  29806 \textsc{Brest} cedex\\\textsc{France}\\Tel +33 (0)2 98 34 88 00\\ \url{www.ensta-bretagne.fr}}
\logoEcole{\includegraphics[height=4.2cm]{logo_ENSTA_Bretagne_Vertical_CMJN}}


\begin{document}
% creer le titre ici
\maketitle
%\include{myabstract}
\tableofcontents

%============= introduction
\section*{Introduction}
%============= Une presentation du contexte
\section{Context of the project}
%============= Une problematisation du theme, des objectifs et des enjeux
%\include{Goal of the project}
%============= Presentation of the market of Android and the market of different level API
%\include{reflection}
%============= Une analyse économique de l'entreprise
%============= Une analyse de l'apport du stage pour le projet professionnel de l'etudient
%\include{conclusion}

%============= Conclusion
%============= annexe
\newpage
\section{Reference}

Market of Android\url{http://www.eco-conscient.com/tag/part-de-marche-android}

\newpage
%\input{appendices}

\end{document}

%%% Local Variables: 
%%% mode: latex
%%% TeX-master: t
%%% End: 